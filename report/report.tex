\documentclass{article}

\usepackage[margin=0.5in]{geometry}
\usepackage{natbib}

\begin{document}

\noindent \large\textbf{A Mulit-Armed Analysis of Policy Space Repeating Oracle
(PSRO)}

\noindent \large Michael Krumdick, Geoffrey Liu

\section{Introduction}
Most Multi-agent reinforcement learning (MARL) can be split up into two
distinct steps \cite{PSRO}. First, given our current population of agents,
define the distribution of agents that minimizes exploitability using a
meta-solver, such as regret minimization. Second, given this distribution, find
a new agent that best exploits it, and add it to the population. This simple
framework encompasses a wide range of different algorithms, including
independent reinforcement learning, self-play, fictitious self-play, and
iterated best response. 

We want to see if we can use what we have learned in class to model the policy
selection step as a multi-armed bandit problem. We hope that this will give us
insight in the following two areas.

\begin{enumerate}
	\item \textbf{Space Complexity:} The meta-solvers are typically complex and
        require computing an $N$ by $N$ payoff matrix for a population of size
        $N$. Using concepts from streaming bandits, 
	\item \textbf{Bounds:} Not much is known about the bounds on sample
        complexity for these types of algorithms. There exist convergence
        guarantees, but nothing guarenteeing how many updates you need to get
        good performance. If we can frame the policy selection step as a
        multi-armed bandit 
\end{enumerate}

Unrelated to multi-armed bandit problems, we are also interested in;
\begin{enumerate}
	\item \textbf{Populations of Regret Based Agents:} Independent RL works as an approximate best
        response, but there exist deep regret based models. To the best of our
        knowledge, no one has tried combining regret based models with a
        PSRO-like approach.
\end{enumerate}

\section{Motivation}
The multi-agent setting for RL agents is the most realistic for many real world
tasks. The environment is often always non-static, meaning the other agents in
the world will be adapting to the agents behavior. There has been a recent push
to focus research on these areas

\section{Plan}
Our plan includes an ambitious goal with a few different "off-ramps" if we get
stuck. The idea is to move in a progression, starting with analysis and
increasing complexity until hopefully we have something novel. We will start be
identifying and environment or series of environments. Ideally, we want a single
environment that can scale in complexity. This will allow us to 

Our hope is to be able to follow the general framework explored in \cite{dip}.
\begin{enumerate}
	\item \textbf{Exploration:} Implement and analyze baselines in simple
        normal-form games. This gives a quick 
	\item \textbf{Design:} Using ideas from the material in this class and our
		understanding of reinforcement learning, try to make a modification or
		adaptation of one of these algorithms that leads to improving some
		quality of performance. This may include sample complexity, space complexity or
		simplicity.
    \item \textbf{Application:} Modify this algorithm to work with deep reinforcement learning agents
      and demonstrate its application on a more complex environment.
\end{enumerate}

One that is particularly interesting is Erdos-Selfrdige-Spencer \cite{ESS} games.

\section{Related Works}
We already have the PSRO \cite{PSRO} framework for representing these types of
algorithms. 

\bibliography{report}

\bibliographystyle{plain}

\end{document}
